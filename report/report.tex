\documentclass[twoside]{article}

%\usepackage{graphics}
\usepackage{graphicx}


\usepackage[sc]{mathpazo} % Use the Palatino font
\usepackage[T1]{fontenc} % Use 8-bit encoding that has 256 glyphs
\linespread{1.05} % Line spacing - Palatino needs more space between lines
\usepackage{microtype} % Slightly tweak font spacing for aesthetics

\usepackage[hmarginratio=1:1,top=32mm,columnsep=20pt]{geometry} % Document margins
\usepackage{multicol} % Used for the two-column layout of the document
%\usepackage{nonfloat}
\usepackage{float}
\usepackage{tikz}
\usepackage{amsmath}

\usepackage[hang, small,labelfont=bf,up,textfont=it,up]{caption} % Custom captions under/above floats in tables or figures
\usepackage{booktabs} % Horizontal rules in tables
\usepackage{float} % Required for tables and figures in the multi-column environment - they need to be placed in specific locations with the [H] (e.g. \begin{table}[H])
\usepackage{hyperref} % For hyperlinks in the PDF

\usepackage{lettrine} % The lettrine is the first enlarged letter at the beginning of the text
\usepackage{paralist} % Used for the compactitem environment which makes bullet points with less space between them

\usepackage{makecell} %table cells
\usepackage{booktabs} %table appearance

\usepackage{abstract} % Allows abstract customization
\renewcommand{\abstractnamefont}{\normalfont\bfseries} % Set the "Abstract" text to bold
\renewcommand{\abstracttextfont}{\normalfont\small\itshape} % Set the abstract itself to small italic text

\usepackage{titlesec} % Allows customization of titles
\usepackage[utf8]{inputenc}

\usepackage{subfig}


\renewcommand{\thesection}{\Roman{section}} 
\renewcommand{\thesubsection}{\thesection.\Roman{subsection}}


\titleformat{\section}[block]{\large\scshape\centering}{\thesection.}{1em}{} % Change the look of the section titles
\titleformat{\subsection}[block]{\large}{\thesubsection.}{1em}{} % Change the look of the section titles


%%%%%%%%%%%%%%%%%%
%%% Table Box 

\def\boxit#1#2{%
	\smash{\color{#1}\fboxrule=1pt\relax\fboxsep=2pt\relax%
		\llap{\rlap{\fbox{\vphantom{0}\makebox[#2]{}}}~}}\ignorespaces
}

%% NEW COLUMN
\newcolumntype{P}[1]{>{\centering\arraybackslash}p{#1}}


%----------------------------------------------------------------------------------------
%	TITLE SECTION
%----------------------------------------------------------------------------------------

\title{\vspace{-15mm}\fontsize{24pt}{10pt}\selectfont\textbf{Classification of the UCI Wine Quality Data Set}} % Article title 

\author{
	\large
	\textsc{Massimiliano Pronesti}\\[2mm] % Your name
	\normalsize Politecnico di Torino \\ % Your institution
	\normalsize \href{mailto:s287646@studenti.polito.it}{s287646@studenti.polito.it} % Your email address
	\vspace{-5mm}
}
\date{}

%----------------------------------------------------------------------------------------

\begin{document}
	
	\maketitle % Insert title
	\begin{abstract}
    \noindent This report provides an analysis of the effectiveness of different classification approaches applied to the popular wine quality prediction problem on the wine dataset from the UCI repository. Specifically, the goal is to predict whether the wine has a good or a bad quality. The original dataset consists of 10 classes. Nevertheless, for this work, the dataset has been binarized, collecting all wines with low quality (lower than 6) into class 0, and good quality (greater than 6) into class 1, while those with quality 6 have been discarded.
    In addition, the dataset contains both red and white wines (merged for the sake of this analysis). There are 11 features, that represent physical properties of the wine, with partially balanced classes.
\end{abstract}

	\begin{multicols}{2} % Two-column layout throughout the main article text
		\section{Data Analysis}

\lettrine[nindent=0em,lines=3]{I}n this section, we are going to conduce an analysis on the main characteristics of the features contained in the training dataset. The training set consists of 1126 bad quality samples and 613 good quality samples, then one class is twice as much present as the other. A visualization of how raw features are distributed is shown in Figure \ref{fig:raw}.

\begin{figure}[H]
	\foreach \i in {0,1,...,10}{
		\includegraphics[width=.15\textwidth]{assets/raw_hist\i}
	}
\caption{Raw features distribution of the UCI Wine Quality Dataset}
\label{fig:raw}
\end{figure}

We can observe that features don't have a zero mean, therefore we might consider standardizing them, i.e. centering data and scaling it by the variance, so that the obtained random variable has zero mean and unitary variance. 

In addition, features don't expose a Gaussian trend, with the presence of some outliers.

Therefore, we gaussianize the features, computing the cumulative rank $r(x)$ over the training set 
\begin{align*}
	r(x) = \frac{\sum^N_{i=1} I[x < x_i] + 1}{N + 2}
\end{align*}

being $x_i$ the value of the considered feature for the i-th sample, and transforming the features computing the inverse of the cumulative distribution function $\Phi$ fed with the rank $r(x)$

\begin{align*}
	X_{gauss} = \Phi^{-1} (r(x)) 
\end{align*}

The distribution of the gaussianized features is shown in Figure \ref{fig:gauss}.


Moreover, we provide an analysis of the correlation between the features, exploiting the Pearson product-moment correlation coefficient, defined as

\begin{align*}
	\rho_{X, Y} = \frac{cov(X, Y)} {\sigma_X \sigma_Y}
\end{align*}


\begin{figure}[H]
	\foreach \i in {0,1,...,10}{
		\includegraphics[width=.15\textwidth]{assets/gauss_hist\i}
	}
	\caption{Gaussianized features distribution of the UCI Wine Quality Dataset}
	\label{fig:gauss}
\end{figure}

being $cov(X, Y)$ the covariance matrix of $X$ and $Y$, expressible as the expectation of the product of $X$ and $Y$ centered using their respective mean.

\begin{align*}
	cov(X, Y) = \mathbb{E}[(X - \mu_X)(Y - \mu_Y)]
\end{align*}

The obtained heatmaps among the gaussianized features are shown in Figure \ref{fig:heat}.

\begin{figure}[H]
	\begin{minipage}{\textwidth}
	\hspace{-.2cm}
	\includegraphics[width=.17\textwidth]{assets/gauss_feat_heat}
	\includegraphics[width=.17\textwidth]{assets/gauss_feat_heat0}
	\includegraphics[width=.17\textwidth]{assets/gauss_feat_heat1}
	\end{minipage}

	\caption{Heatmaps among Gaussianized features}
	\label{fig:heat}	
\end{figure}

where we can observe some features are correlated (darker colors in the heatmap), thus exploring dimensionality reduction techniques such as the PCA, could prove beneficial.
		\section{Method}
In this section we are going to first describe the approach followed towards model selection and model evaluation. Then, we will analyse the explored classification methods and the results they yielded.
 
\subsection{Approach}
In order to perform our analysis, we will adopt two approaches towards model selection:
\begin{itemize}
	\item single split: the training set is divided into two chunks, where the 80\% of the samples are used for fitting the classifer and the remaining 20\% for testing it. 
	\item k-fold cross validation: the training set is split into $k$ folds, one of whom is used of testing and the other $k - 1$ for fitting the model. The process is repeated $k$ times. This approach usually makes the process of model selection more reliable as, one by one, all the chunks will be used as unseen data. 
	For this specific application, we set $k = 5$, i.e. we used 5 folds.
\end{itemize}
In both cases, we make sure no transformation is applied on the whole training data before splitting it, not to introduce data leakages.  

As regards model evalution, we want to be Bayesian and adopt the minimum of the normalized Bayesian risk as metric, which measures the cost we would pay if we made optimal decisions using the recognizer scores. The application of interest is a uniform prior one

\begin{align*}
	(\tilde{\pi}, C_{fp}, C_{fn}) = (0.5, 1, 1)
\end{align*}

being $\pi$ the (unbiased, in the specified case) prior, $C_{fp}, C_{fn}$ the cost of the false positive and false negative case, respectively.

%%%%%%%%%%%%%%%%%%%%%%%%%%%%%%%%%%%%%%%%%%%%
%% Gaussian Classifiers
%%%%%%%%%%%%%%%%%%%%%%%%%%%%%%%%%%%%%%%%%%%%
\subsection{Gaussian Classifiers}

Gaussian classifiers are the first class of methods we take into considerations. In particular we analyse the performances yielded by a Multivariate Gaussian and a Naive Bayes, both with full and tied covariance, for a total of 4 classifiers.
 
Table \ref{tab:gaus_res} shows the results obtained for these models both for the single split and for the k-fold cross validation.

\noindent
\begin{table}[H]
\begin{tabular}{ p{2.5cm} p{2cm} p{1.5cm}  }
	\hline
	& \makecell{\textbf{Single split} \\ $\tilde{\pi} = 0.5$} & \makecell{\textbf{5-fold} \\ $\tilde{\pi} = 0.5$} \\
	\hline
	\multicolumn{3}{c}{Raw features} \\
	\hline
	Full-Cov & 0.00 &  0.00 \\
	Diag-Cov & 0.00 & 0.00 \\
	Tied Full-Cov & 0.00 & 0.00 \\
	Tied Diag-Cov &  0.00 & 0.00 \\	
	\hline
	\multicolumn{3}{c}{Gaussianized features} \\
	\hline
	Full-Cov & 0.00 &  0.00 \\
	Diag-Cov & 0.00 & 0.00 \\
	Tied Full-Cov & 0.00 & 0.00 \\
	Tied Diag-Cov &  0.00 & 0.00 \\	
	\hline
	\multicolumn{3}{c}{Gaussianized features, PCA(m=10)} \\
	\hline
	Full-Cov & 0.00 &  0.00 \\
	Diag-Cov & 0.00 & 0.00 \\
	Tied Full-Cov & 0.00 & 0.00 \\
	Tied Diag-Cov &  0.00 & 0.00 \\	
	\hline
	\multicolumn{3}{c}{Gaussianized features, PCA(m=9)} \\
	\hline
	Full-Cov & 0.00 &  0.00 \\
	Diag-Cov & 0.00 & 0.00 \\
	Tied Full-Cov & 0.00 & 0.00 \\
	Tied Diag-Cov &  0.00 & 0.00 \\	
	\hline
\end{tabular}
\caption{min DFC for Gaussian models}
\label{tab:gaus_res}
\end{table}


%%%%%%%%%%%%%%%%%%%%%%%%%%%%%%%%%%%%%%%%%%%%
%% 		Logistic Regression
%%%%%%%%%%%%%%%%%%%%%%%%%%%%%%%%%%%%%%%%%%%%
\subsection{Logistic Regression}

In this section, we assess the performances of a Logistic Regression classifier, both with linear and quadratic kernel.

Being classes unbalanced, their costs are rebalanced using a loss function accounting for the number of samples for each class, used as weights for the two terms deriving from the split of the summation of the original loss

\begin{align*}
	J(w, b) &= \cfrac{\lambda}{2} \|w\|^2 + \cfrac{\pi_T}{n_T} \sum_{i|c_i=1} log \ \sigma(y_i)^{-1} 
	\\ &+ \cfrac{1 - \pi_T}{n_T} \sum^n_{i| c_i=0} log \ \sigma(y_i)^{-1} 
\end{align*}
being $\sigma$ the sigmoid function, $y_i$ the output of the model $y_i = - z_i (w^T x_i + b)$, $z_i$ a variable equalt to $\pm 1$ depending on the class, $w, b$ the parameters of the model. 



%%%%%%%%%%%%%%%%%%%%%%%%%%%%%%%%%%%%%%%%%%%%
%% 	    Support Vector Classifier
%%%%%%%%%%%%%%%%%%%%%%%%%%%%%%%%%%%%%%%%%%%%

\subsection{Support Vector Classifier}
In this section, we test the performances of 
different flavours of support vector classifiers. In particular, we are going to use a linear kernel, a quadratic polynomial kernel and a radial basis function kernel.

As done for the previous class of models, we will also take into account the possibility of rebalancing the classes introducing an empirical prior $\pi_{T,emp}$ over the training set, thus using two different values of $C$ for the bad and good quality wines

\begin{align*}
	C_T = C \cfrac{\pi_T}{\pi_{T,emp}} \ \ \ \ \ \
	C_F = C \cfrac{1 - \pi_T}{1 - \pi_{T,emp}} 
\end{align*}



%%%%%%%%%%%%%%%%%%%%%%%%%%%%%%%%%%%%%%%%%%%%
%% 		Gaussian Mixtures
%%%%%%%%%%%%%%%%%%%%%%%%%%%%%%%%%%%%%%%%%%%%
\subsection{Gaussian Mixture Models}
Eventually, Gaussian Mixutres are the last type of classifier employed for this task, for which we expect to yield generally better results than Gaussian models, as GMMs can approximate generic distributions.

We consider both full and diagonal covariance models, with and without covariance tying, where tying takes place at class level (i.e. different classes have different covariance matrices).
		\section{Calibration}\label{sec_calib}
So far, we only considered the minimum detection cost as a metric to evaluate the different models. However, the cost we pay depends on the threshold we use to perform the class assignments. In this section, we aim at assessing the performances of the best selected classifiers using the optimal theoretical threshold $t_{opt} = - log \frac{\tilde{\pi}}{1-\tilde{\pi}}$ as well as assessing the performances after calibrating the scores and after fusing models.

\subsection{Scores on different thresholds}
The first analysis we conduct refers to the detection cost function using as threshold the optimal theoretical one. This metric is referred to as the \textbf{actual DCF}.

 The obtained results applying $t_{opt}$ are reported in Table \ref{tab:calib}, where we notice a slight worsening of the performances, which is compatible with the fact that the tested models are non-probabilistic, which implies they do not account for variations in the data, leading to uncalibrated scores.

\noindent
\begin{table}[H]
	\resizebox{.5\textwidth}{!}{
		\begin{tabular}{ P{4cm} P{2cm} P{3cm} }
			\hline
			\hline
			&  \makecell{\textbf{5-fold} $(\tilde{\pi} = 0.5)$}\\
			\hline	
			& min DCF & act DCF $(t_{opt})$ \\	
			\hline
			RBF SVC  & 0.215 & 0.234 \\
			Quad LR  & 0.273 & 0.286  \\	
			GMM      & 0.280 & 0.298  \\
			\hline
		\end{tabular}
	}
	\caption{minimum and actual DCF for the best performing models}
	\label{tab:calib}
\end{table}

To tackle this problem we will employ a twofold approach: on the one hand, we will compute a threshold $t^*$ for each application, i.e. for each fold we select the threshold that gives the minimum DCF for an
application on the validation set; on the other hand, we will choose a transformation function performing the mapping $s \to s_{cal} = f(s)$, being $f(\cdot)$ a linear function

\begin{align*}
	f(s) = \alpha s + \beta \underbrace{- log \tfrac{\tilde{\pi}}{1- \tilde{\pi}}}_{t_{opt}}
\end{align*}

where $\alpha$ and $\beta$ are estimated via a Linear Logistic Regression. In fact, since the score of the Linear Logistic Regression acts as a posterior log-likelihood ratio, we will recover the calibrated score just subctracting the theoretical optimal threshold $t_{opt}$. We apply this approach trying different values of $\lambda$ and eventually picking the best one for each scenario.

The calibrated scores are report in Table \ref{tab:calib2}


\noindent
\begin{table}[H]
	\resizebox{.5\textwidth}{!}{
		\begin{tabular}{ P{4cm} P{2cm} P{3cm} }
			\hline
			\hline
			&  \makecell{\textbf{5-fold} $(\tilde{\pi} = 0.5)$}\\
			\hline
			& act DCF ($t^*$) & calibrated (LR)	\\	
			\hline
			RBF SVC  &  0.223 & 0.229\\
			Quad LR  &  0.284 & 0.289 \\	
			GMM      &  0.292 & 0.301 \\
			\hline
		\end{tabular}
	}
	\caption{actual estimated and calibrated DCF for the best performing models}
	\label{tab:calib2}
\end{table}

We notice that the estimated threshold improves the scores, while the calibration with the LR model proves ineffective, as we experience a worsening of the results. 


\subsection{Combining the best classifiers}
In this section, we analyze the performances of the chosen models, in terms of min and actual DCF, when they're fused, i.e. we combine the scores yielded by the different classifier as follows

\begin{align*}
	S = w^T s + b
\end{align*}

being $s$ the array of scores and $w, b$ the parameters of a Linear Logistic Regression. We again do a sweep on different values of $\lambda$ and operate using a 5-fold cross validation approach. The "fusions" taken into account are those involving the RBF SVC (the best performing model we have), i.e. we discard the fusion between the Quadratic Logistic Regression an the Gaussian Mixture. Table \ref{tab:calib_fus} reports the obtained calibrated scores.

\noindent
\begin{table}[H]
	\resizebox{.5\textwidth}{!}{
		\begin{tabular}{ P{3.5cm} P{2cm} P{3cm} }
			\hline
			\hline
			&  \makecell{\textbf{5-fold} $(\tilde{\pi} = 0.5)$}\\
			\hline
			& min DCF & act DCF ($t_{opt}$)\\	
			\hline
			SVC + QLR       & \boxit{red}{.35in}0.214 & \boxit{red}{.35in}0.219 \\
			SVC + GMM  	    & 0.220 & 0.231 \\	
			SVC + GMM + QLR & \boxit{cyan}{.35in}0.215 & 0.222 \\
			\hline
		\end{tabular}
	}
	\caption{minimum and actual DCF for the combination of best performing models}
	\label{tab:calib_fus}
\end{table}

We notice that the min DCF improves w.r.t. all single models but the RBF Support Vector Classifier, whose performances are as good as the fusion scenario \footnote{The fusion between RBF SVC and Quad LR is unnoticeably better, at the expense of a more complex model}.   

\subsection{Discussion}
From the above results, we conclude that the fusion of the RBF SVC with the Quadratic Logistic Regression yields similar results of the same models fused with an 8 components full-covariance Gaussian Mixture Model. Our choice is, thereby, the simplest of the two as, given the same performances, the simpler of the two reduces the odds of incurring into overfitting and introduces, overall, less computational overheads. 
		\section{Evaluation}

\begin{table}[H]
	\caption{Algunos equipos}
	\centering
	\begin{tabular}{llr}
		\toprule
		\multicolumn{2}{c}{Barómetro} \\
		\cmidrule(r){1-2}
		Equipo & Rango hPa& Precio  \\
		\midrule
		
		Baromería en hPa & 580 - 1040& $998USD$ \\\\
		Registrador bar & 10 – 999.9 & $218USD$ \\\\
		BARÓMET AB60 & 800-1100&$$\\\\
		BARÓMET AB100 &600-1100&\\\\
		VAISALA PTB110 &5,6,8,11 (x100)&\\\\
		\bottomrule
	\end{tabular}
\end{table}

Aquí presentamos mayor detalles de los equipos de la tabla de arriba:\\\\

Barómetro AB 60\\
Fabricante: Ammonit\\
Venta en el interior del armario de conexiones y por separado\\
Sensor de presión piezoeléctrico\\
Intérvalo de medida 800-1100 hPa (mbar)\\
Tensión de salida: 0-5 VDC\\Tensión entrada: 9-32 V\\
Consumo: < 5 mA @ 12 VDC\\
Temperatura de funcionamiento: -40 - 85 C\\
Humedad de funcionamiento: 0-98 po ciento RH\\
Atmósfera: No iónica, no corrosiva\\
Tiempo de respuesta: 10-90 por ciento, typ. 50 ms.\\\\

VAISALA PTB 110\\
Fabricante: Vaisala\\
Intérvalo de medida: 500, 600, 800-1100 hPA\\
Sensor piezoeléctrico: Bajo consumo\\
Medida de diferentes intérvalos
Exactitud:tol.:3 hPa a 20 C;tol.:1 hPa at -20-60 C tol.:1,5 hPa at -40-60 C\\\\
Tensión de salida: 0-2,5 ó 0-5 VDC\\
Consumo: < 4 mA @ 12 VDC\\
		\section{Conclusion}
In this work we analyzed the UCI Wine Quality Dataset applying a set of "traditional" Machine Learning Models and using the minum detection cost function and its derivatives as metrics. We showed how linear models are not very well suited for classifying wine features and how other type of space transformations adapt better to the data. We eventually explored the benefits of combining the most performing models and proved how our approach is effective on test data, yielding results comparable to the expected ones in light of the analysis conducted towards model selection and tuning.





	\end{multicols}
	
\end{document}