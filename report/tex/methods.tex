		\section{Methods}
Actualmente existen varios tipos de Barómetros. Según los fines del instituto o experimento a realizarse uno es ocupado o no, así mismo, la tecnología y el presupueto disponible, varía su construcción y manufactura. A continuación los nombramos:\\


\begin{compactitem}
	\item Barómetro de Mercurio:
	Está formado por un tubo de vidrio de unos 850 mm de altura, cerrado por elextremo superior y abierto por el inferior. Cuando el tubo se llena de mercurio yse coloca el extremo abierto en un recipiente lleno del mismo líquido, el niveldel tubo cae hasta una altura de unos 760 mm por encima del nivel delrecipiente y deja un vacío casi perfecto en la parte superior del tubo, estando alnivel del mar y en condiciones normales. Las variaciones de la presiónatmosférica hacen que el líquido del tubo suba o baje ligeramente. La lecturade un barómetro de mercurio puede tener una precisión de hasta 0,1 mm.\\
	
	
	\item Barómetro de Fortín:
	Compuesto por un tubo de Torricelli que se introduce en el mercurio contenido en una cubeta de vidrio en forma tubular, provista de una base de piel de gamo cuya forma puede ser modificada por medio de un tornillo que se apoya en su centro y que, oportunamente girado, lleva el nivel del mercurio del cilindro a rozar la punta de un pequeño cono de marfil. Así se mantiene un nivel fijo.Este está completamente recubierto de latón, salvo dos ranuras verticales junto al tubo que permiten ver el nivel de mercurio. En la ranura frontal hay una graduación en milímetros y una escala de vernier (nonio) para la lectura de décimas de milímetros.Y en la posterior hay un pequeño espejo para facilitar la visibilidad del nivel.Los barómetros Fortin se usan en laboratorios científicos para las medidas de alta precisión.\\
	
	
	\item Barómetro Aneroide:
	Es un barómetro que no utiliza mercurio. Indica las variaciones de presión atmosférica por las deformaciones más o menos grandes que aquélla hace experimentar a una caja metálica de paredes muy elásticas en cuyo interior se ha hecho el vacío más absoluto. Se gradúa por comparación con un barómetro de mercurio pero sus indicaciones son cada vez más inexactas por causa de la variación de la elasticidad del resorte plástico. Fue inventado por Lucien Vidie en 1843 y es más grande, por lo tanto el barómetro que utiliza mercurio. El principio de funcionamiento es el cambio de tamaño de una cápsula parcialmente evacuada, construida para maximizar el cambio en una dimensión con los cambios en la presión del aire. Los cambios de la longitud de la cápsula se amplifican por medio de un brazo a un dial, que permite mostrar la presión.\\
	\
	\item Barómetro Holostérico 
	Está formado por un recipiente aplanado, de superficies onduladas en el que se ha logrado una intensa rarefacción antes de cerrarlo; en una de las caras se apoya un resorte que, con las variaciones de presión atmosférica, hace mover un índice por medio de un juego de palancas.Es menos preciso que el Aneroide.
	
	
	
\end{compactitem}